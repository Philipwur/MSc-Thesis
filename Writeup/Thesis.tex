\documentclass[conference]{IEEEtran}
\IEEEoverridecommandlockouts
% The preceding line is only needed to identify funding in the first footnote. If that is unneeded, please comment it out.
\usepackage{cite}
\usepackage{amsmath,amssymb,amsfonts}
\usepackage{algorithmic}
\usepackage{graphicx}
\usepackage{textcomp}
\usepackage{xcolor}

\def\BibTeX{{\rm B\kern-.05em{\sc i\kern-.025em b}\kern-.08em
    T\kern-.1667em\lower.7ex\hbox{E}\kern-.125emX}}
\begin{document}

\title{Optimisatimising the Crystal Structure of a Novel Metamaterial with
 Negative Electric Susceptibility\\
\thanks{Identify applicable funding agency here. If none, delete this.}
}



\author{\IEEEauthorblockN{Philip Daniel Würzner}
\IEEEauthorblockA{\textit{School of Electronic Engineering and Computer Science} \\
\textit{Queen Mary University of London}\\
London, United Kingdom}
\and
\IEEEauthorblockN{Supervisor:}
\IEEEauthorblockA{\textit{Flynn Castles} \\
\and
\IEEEauthorblockN{Co-Supervisor:}
\IEEEauthorblockA{\textit{Rahul Dutta} \\
}
}
}

\maketitle

\begin{abstract}
the abstract of the document goes here
\end{abstract}

\begin{IEEEkeywords}
Metamaterial, Optimisation, Negative Electric Susceptibility
\end{IEEEkeywords}

\section{Introduction}
Mention metamaterials here, the potential the project has and the role of 
optimisation in science.

\section{Related Work}

\subsection{Electric Susceptibility}

Electric Susceptibility is the

\subsection{Ferroelectricity and The Claussius-Mossotti Model of Spontaneous 
Polarisability}

aaaaaaaaaaaaaaa

\subsection{Crystal Lattices}

mention bravais 2D Lattices

\subsection{Optimisation Algorithms}

aaaaaaaaaaaaaaa


\section{Methodology}

go through the program\\
- Discuss Rahul's Method Here\\
- Discuss optimisation pipeline\\
- Discuss the workflow of the script \\
- Discuss Optimisation Variables (i.e. bounds)\\

\section{Simulations and Results}

go over performance, results and optimisation, error estimation

\section{Conclusion}

conclude here

\section*{Acknowledgment}

fanks guys


\begin{thebibliography}{00}
\bibitem{b1} G. Eason, B. Noble, and I. N. Sneddon, ``On certain integrals of Lipschitz-Hankel type involving products of Bessel functions,'' Phil. Trans. Roy. Soc. London, vol. A247, pp. 529--551, April 1955.
\bibitem{b2} J. Clerk Maxwell, A Treatise on Electricity and Magnetism, 3rd ed., vol. 2. Oxford: Clarendon, 1892, pp.68--73.
\bibitem{b3} I. S. Jacobs and C. P. Bean, ``Fine particles, thin films and exchange anisotropy,'' in Magnetism, vol. III, G. T. Rado and H. Suhl, Eds. New York: Academic, 1963, pp. 271--350.
\bibitem{b4} K. Elissa, ``Title of paper if known,'' unpublished.
\bibitem{b5} R. Nicole, ``Title of paper with only first word capitalized,'' J. Name Stand. Abbrev., in press.
\bibitem{b6} Y. Yorozu, M. Hirano, K. Oka, and Y. Tagawa, ``Electron spectroscopy studies on magneto-optical media and plastic substrate interface,'' IEEE Transl. J. Magn. Japan, vol. 2, pp. 740--741, August 1987 [Digests 9th Annual Conf. Magnetics Japan, p. 301, 1982].
\bibitem{b7} M. Young, The Technical Writer's Handbook. Mill Valley, CA: University Science, 1989.
\end{thebibliography}
\vspace{12pt}

\end{document}
